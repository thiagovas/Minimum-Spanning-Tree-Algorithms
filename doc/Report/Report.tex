\documentclass[11pt]{article}
\usepackage[utf8]{inputenc}
\usepackage[brazil]{babel}
\title{\textbf{Um Estudo dos Algoritmos de Árvore Geradora Mínima}}
\author{Lucas Assunção\\Thiago Vieira}
\date{}
\begin{document}

\maketitle

\section{Introdução}
Neste trabalho iremos estudar o problema da árvore geradora mínima, mostrar os diferentes algoritmos para resolver o problema e uma formulação multifluxo de programação linear.\\
% TODO: Falar mais coisas aqui.


\section{Problema}
Dado um grafo não direcionado G(V, E) e arestas E(U, V, W) ligando U a V com pesos reais W, o problema da árvore geradora mínima busca um subgrafo conexo G'
% TODO: Explicar o problema da arvore geradora minima.

    \subsection{Aplicações}



\section{Algoritmos}
%TODO: Mostrar teoremas.
%TODO: Prova de corretude.
%TODO: Em cada subsection falar sobre os detalhes de implementação de cada variante.

    \subsection{Kruskal}
    
    \subsection{Prim}
    
    \subsection{Borůvka}
        \subsubsection{Iterativo}
                
        \subsubsection{Paralelo}



\section{Resultados Experimentais}



\section{Conclusão}





% PROPOSAL
Para o trabalho de teoria dos grafos, iremos implementar algoritmos que calculam a árvore geradora mínima de um grafo, explicar como funciona cada um e comparar seus resultados para algumas classes de grafos. Segue os algoritmos que analisaremos:\\
\begin{enumerate}
\item Kruskal
\item Prim
\item Borůvka, versão sequencial iterativa
\item Borůvka, versão em paralelo
\end{enumerate}


\end{document}
